\phantomsection
\setsection{Chương 5: Kết luận và hướng phát triển}
\setcounter{section}{5}
\setcounter{subsection}{0}

\phantomsection

\subsection{Kết quả đạt được}
Về mặt thực tiễn: Đồ án đã giải quyết thách thức ban đầu bằng cách cung cấp một giải pháp cho việc quản lý và theo dõi mượn sách, mang lại thuận lợi cho người quản lý và đọc giả trong việc tìm kiếm sách.
\par
Về mặt nội dung: Đề tài "Website cửa hàng sách" đã được thiết kế với các chức năng linh hoạt để đáp ứng nhu cầu sử dụng của người dùng. Sử dụng công nghệ Mongoose để quản lý dữ liệu, Bootstrap 5 và Ant Design để tạo giao diện thân thiện và hấp dẫn hơn, kết hợp JWT để bảo vệ tính xác thực của người dùng, cùng với các công nghệ khác để tạo ra một trang web hoàn chỉnh.
\par
Về chức năng: Tất cả các chức năng cơ bản đã được hoàn thiện, đáp ứng đầy đủ nhu cầu của người dùng.


\subsection{Hạn chế}
Do thời gian thực hiện đề tài hạn chế, không có đủ thời gian để thêm một số chức năng và tùy chỉnh để cải thiện trải nghiệm người dùng, bao gồm chức năng quên mật khẩu và tùy chỉnh giao diện như chế độ sáng/tối và đa ngôn ngữ.


\phantomsection

\subsection{Hướng phát triển}
Để tối ưu hóa trải nghiệm của người dùng và tăng tính hiệu quả của trang web, có một số hướng phát triển tiềm năng như sau:

\begin{itemize}[label={+}]
          \item Thêm chức năng mua sách, quản lý hóa đơn và phương thức thanh toán: Đáp ứng nhu cầu của người dùng muốn mua sách để tiện lợi hơn trong việc đọc sách.
          \item Phân loại sách: Thêm tính năng phân loại sách theo thể loại trong cơ sở dữ liệu, giúp người dùng tìm kiếm và lựa chọn sách dễ dàng hơn theo thể loại và giá tiền.
          \item Tùy chỉnh giao diện: Cung cấp khả năng tùy chỉnh giao diện cho người dùng, bao gồm chế độ sáng/tối và đa ngôn ngữ, nhằm tối ưu hóa trải nghiệm người dùng.
          \item Thêm chức năng thống kê cho trang quản lý: Cung cấp các báo cáo và thống kê liên quan đến hoạt động của trang web để người quản lý có thể đánh giá hiệu suất và tối ưu hóa quản lý.
          \item Thêm phần đánh giá cho mỗi quyển sách: Cho phép người đọc đánh giá và chia sẻ ý kiến về các quyển sách, giúp người dùng có cái nhìn tổng quan về chất lượng và sự phong phú của nội dung sách.
        \end{itemize}