\phantomsection
\setsection{Chương 1: Giới thiệu}
\setcounter{section}{1}

\phantomsection
\subsection{Đặt vấn đề}
Trong thế giới hiện đại đang phát triển với tốc độ chóng mặt của công nghệ thông tin và phần mềm, việc áp dụng chúng vào mọi lĩnh vực của cuộc sống không chỉ là một xu hướng mà còn là một bước tiến quan trọng để nâng cao hiệu quả và tiện ích cho cả xã hội. Từ việc quản lý thông tin cá nhân đến quản lý doanh nghiệp, xu hướng thương mại điện tử tăng cao,  công nghệ thông tin đang trở thành trái tim của mọi hoạt động.\par
Trong lĩnh vực quản lý cho thuê mượn sách, việc tận dụng công nghệ để xây dựng một hệ thống quản lý thông minh và hiệu quả là điều không thể phủ nhận. Điều này không chỉ giúp cho việc quản lý sách và độc giả trở nên dễ dàng hơn mà còn mở ra những cánh cửa mới cho trải nghiệm người dùng.\par
Bằng cách tạo ra trang web quản lý mượn sách và trang web mượn sách trực tuyến cho đọc giả, chúng ta không chỉ đơn thuần là tạo ra một công cụ quản lý mà còn là một nguồn cảm hứng và khích lệ cho sự phát triển của văn hóa đọc. Tính năng quản lý tài khoản, nhà xuất bản, sách, theo dõi mượn sách của tác giả, giúp người thủ thư có thể dễ dàng theo dõi và quản lý việc mượn sách của đọc giả. Đọc giả cũng có thể dễ dàng thực hiện các thao tác mượn sách và theo dõi lịch sử mượn sách của mình.\par
So với việc quản lý mượn sách và đăng ký mượn sách truyền thống, việc xây dựng một hệ thống quản lý mượn sách thông minh không chỉ là một cách tiết kiệm thời gian và chi phí mà còn là một cơ hội để tạo ra một cộng đồng đam mê sách phát triển và sôi động hơn. Điều này đồng nghĩa với việc mở ra những cánh cửa mới cho việc truyền bá tri thức và phát triển cá nhân.


\phantomsection
\subsection{Mục tiêu đề tài}
Đầu tiên, Tạo ra một website giúp các nhân viên cửa hàng sách, hoặc thư viện quản lý việc mượn sách của các đọc giả một cách dễ dàng, tiện lợi 
.\par
\begin{itemize}[align=left, leftmargin=2cm]
  \item[\textbf{-- Về giao diện và trải nghiệm người dùng (UI/UX)}]:
Giao diện đơn giản, hiện đại với màu trắng xanh là chủ đạo. Thiết kế phù hợp công việc quản lý. Đem lại trải nghiệm tuyệt vời cho người quản lý với việc thực hiện các theo dõi mượn sách, quản lý tài khoản đọc giả và quản lý sách thuận lợi với các thao tác đơn giản dễ dàng sử dụng và các hiệu ứng chuyển trang mượt mà đem lại cảm giác dễ chịu và thoải mái cho người dùng \par
  \item[\textbf{-- Về quản lý}]:
việc xây dựng cơ sở dữ liệu với logic chặt chẽ  giúp tối ưu hóa khả năng quản lý, các thông tin tài khoản người dùng bảo mật chặt chẽ và trách tình trạng rò rõ các dữ liệu quan trọng người dùng. Hệ thống  phân quyền người dùng dựa trên vai trò của họ, trong website quản lý chỉ có người quản lý mới có thể đăng nhập và thực hiện các tác vụ quản lý.  \par
\end{itemize}

Thứ hai, Tạo ra một website bán sách với các tính năng tuyệt vời để các cửa hàng sách hoặc thư viện tiếp cận các đọc giả đồng, đồng thời việc tạo ra một website có thể giúp cửa hàng quảng bá, tiếp thị sách một cách hiệu quả với chi phí duy trì thấp nhưng các tiện ích và tiềm năng là rất lớn cho sự phát triển của cửa hàng và cộng đồng người đọc sách.\par
\begin{itemize}[align=left, leftmargin=2cm]
  \item[\textbf{-- Về giao diện và trải nghiệm người dùng (UI/UX)}]:
giao diện đẹp và hiện đại với màu xanh lục đậm và màu trắng đây là 2 màu sách đem lại sự dễ chịu cho người dùng đồng thời nó cũng toát lên vẻ hiện đại. với bố cục website hài hoài hợp lý giúp người dùng có cái nhìn đặc biệt về website  điều này mang lại hiệu quả cho việc tiếp cận và quảng bá của hàng cũng như tiếp thị các sản phẩm mới đến đọc giả. \par
  \item[\textbf{-- Về hiệu năng hệ thống}]:
việc tối ưu  hệ thống giúp các thao tác người dùng thực hiện trên website có thể diễn ra một cách nhanh chóng tạo cảm giác dễ chịu cho người dùng. Kiểm tra chặt chẽ giúp ngăn chặn việc hệ thống phát sinh các lỗi không đáng có gây cảm giác khó chịu cho người dùng \par
\end{itemize}

\phantomsection
\subsection{Đối tượng và phạm vi nghiên cứu}
Website mượn sách trực tuyến là một trong những nền tảng quan trọng giúp người đọc tiếp cận với hàng ngàn tựa sách một cách thuận tiện và linh hoạt. Đối tượng của nghiên cứu này bao gồm cả người đọc, các thư viện, cửa hàng sách. Mục tiêu của nghiên cứu là tối ưu hóa trải nghiệm người dùng thông qua việc cải thiện giao diện và quy trình mượn sách, đồng thời tối ưu hóa hệ thống quản lý sách để đảm bảo nguồn sách luôn được cập nhật và phân phối một cách hiệu quả. Đọc giả khi có tài khoản thì có thể truy cập trang web thực hiện các tác vụ mượn sách. Riêng nhân viên quản lý với tài khoản riêng của mình có thể hiện các tác vụ quản lý tài khoản, sách, nhà xuất bản và quản lý việc mượn sách\par

\phantomsection
\subsection{Phương pháp nghiên cứu}
\begin{itemize}[label={-}]
    \item Thu nhập các thông tin từ các cửa hàng thương mại trực tuyến sẵn có như shopee, Lazada,.. và cửa hàng bán sách trực tuyến hiện có và sao đó lên ý tưởng.
    \item Tổng hợp các kiến thức về tổ chức, phân tích, thiết kế cơ sở dữ liệu và ngôn ngữ mô hình hóa UML.
    \item Tìm hiểu và sử dụng các công nghệ xây dựng website: vue js,  bootstrap 5, express, mongoose, mongodb,  jwt, ant design.
    \item Nắm vững các kỹ năng lập trình backend, frontend và thiết kế giao diện, thiết kế hệ thống.
\end{itemize}

\phantomsection
\subsection{Các chức năng chính}
Website quản lý cho mượn sách được tạo ra nhằm giúp người quản lý dễ dàng trong việc quản lý việc cho mượn sách, tra cứu được thời gian cho mượn và thời gian trả của từng cuốn sách. \par
Đối với người mượn sách có thể dễ dàng tìm kiếm các cuốn sách hiện có và đặt mượn một cách trực tuyến.

\begin{itemize}[align=left, leftmargin=2cm]
  \item[\textbf{-- Đối với quản trị viên}]:
        \begin{itemize}[label={+}]
          \item Đăng nhập, đăng xuất.
          \item Quản lý tài khoản (Thêm, sửa và xóa tài khoản đọc giả).
          \item Quản lý sách (Thêm, sửa và xóa sách).
          \item Quản lý nhà xuất bản (Thêm, sửa, và xóa nhà xuất bản).
          \item Quản lý mượn sách (Thêm, sửa và xóa lịch sử mượn sách).
        \end{itemize}
  \item[\textbf{-- Đối với đọc giả}]:
        \begin{itemize}[label={+}]
          \item Đăng nhập, đăng xuất, tạo tài khoản.
          \item Xem danh sách sản phẩm.
          \item Quản lý thông tin cá nhân: xem và chỉnh sửa thông tin cá nhân. cho sách)
          \item Tìm kiếm sách.
          \item Xem mô tả chi tiết sách.
          \item Mượn sách.
        \end{itemize}
\end{itemize}
